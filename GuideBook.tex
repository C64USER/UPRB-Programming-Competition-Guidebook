\documentclass[letterpaper]{article}

\usepackage{algorithm}
\usepackage[noend]{algpseudocode}
\usepackage[margin=1in]{geometry}
\usepackage{listings}

\title{Programming competition guidebook}
\author{
  Juan J. Alvarez \\
  juan.alvarez7@upr.edu \\
  UPR Bayam\'{o}n \\
  Computer Science Department
}
\date{}

\begin{document}

  \maketitle

  \tableofcontents

  \newpage

  \section{Mathematics}

    \subsection{Check whether a number is prime}
      In this section we talk about how to programatically check the primality of a given number.

      \subsubsection{Pseudocode}
        \begin{algorithm}
          \caption{Prime check algorithm}
          \begin{algorithmic}[1]
            \Procedure{isPrime}{$n$}
              \If{$n < 2$}
                \Return $false$
              \EndIf
              \If{$n == 2$}
                \Return $true$
              \EndIf
              \If{$n \% 2 == 0$}
                \Return $false$
              \EndIf
              \For{$x = 3$ ; $x^2 <= n$ ; $x += 2$}
                \If{$n \% x == 0$}
                  \Return $false$
                \EndIf
              \EndFor
              \Return $true$
            \EndProcedure
          \end{algorithmic}
        \end{algorithm}

      \subsubsection{Implementation}
        \lstinputlisting{code/isprime.java}

    \subsection{List the divisors of a number}
      In this section we talk about how to programatically list the divisors of a given number.

      \subsubsection{Pseudocode}
        \begin{algorithm}
          \caption{Divisor list algorithm}
          \begin{algorithmic}[1]
            \Procedure{listDivisors}{$n$}
              \State $list$ = new empty list of numbers
              \State add $1$ to $list$
              \State $mpd = \sqrt{n}$
              \For{$x = 2$ ; $x <= mpd$ ; $x++$}
                \If{$n \% x == 0$}
                  \State add $x$ to $list$
                  \If{$n \div x \neq x$}
                    \State add $n/d$ to $list$
                  \EndIf
                \EndIf
              \EndFor
              \State add $n$ to $list$ \\
              \Return $list$
            \EndProcedure
          \end{algorithmic}
        \end{algorithm}

    \subsubsection{Implementation}
      \lstinputlisting{code/listdivisors.java}

    \subsection{Calculating Factorials}

      \subsubsection{Pseudocode}
        \begin{algorithm}
          \caption{Factorial Algorithm}
          \begin{algorithmic}[1]
            \Procedure{factorial}{$n$}
              \State $total=1$
              \For{$x = n$ ; $x > 1$ ; $x++$}
                \State $total*=x$
              \EndFor
              \Return $total$
            \EndProcedure
          \end{algorithmic}
        \end{algorithm}

      \subsubsection{Implementation}
        This method has a huge flaw due to the fact that it uses 64-bit integers to calculate the result, it is not capable of calculating the factorial of any number larger than 16.
        \lstinputlisting{code/factorial1.java}

      \subsubsection{BigInteger Implementation}
        This implementation was done using the BigInteger class to solve the limitational problems of using 64-bit integers.
        \lstinputlisting{code/factorial2.java}

    \subsection{Sum of Natural Numbers}
      The sum of the sequence of natural numbers \{$1, 2, 3, ... , n$\} can be written as \[\sum_{x=1}^{n} x = \frac{n(n+1)}{2}\].

    \subsection{Divisibility Rules}
    \begin{tabular}{ | r | l |}
      \hline
      2 & The last digit is 0, 2, 4, 6 or 8. Or the modulus operation with 2 yields 0. \\
      \hline
      3 & The sum of the digits is divisible by 3. \\
      \hline
      4 & The number formed by the last two digits is divisible by 4. \\
      \hline
      5 & The last digit is either 0 or 5. \\
      \hline
      6 & It is divisible by 2 AND it is divisible by 3. \\
      \hline
      7 & If the last digit multiplied by two and subtracted from the rest of the number is divisible by 7. \\
      \hline
      8 & The last three digits are divisible by 8. \\
      \hline
      9 & The sum of the digits is divisible by 9. \\
      \hline
      10 & The last digit is 0. \\
      \hline
      11 & The difference between the sum of the odd placed digits and the sum of the even placed digits is divisible by 11. \\
      \hline
      12 & The number is divisible by both 3 and 4. \\
      \hline
      13 & Subtract 9 times the last digit from the rest of the number, the result is divisible by 13. \\
      \hline
      14 & It is divisible by 2 and 7. \\
      \hline
      15 & It is divisible by 3 and 5. \\
      \hline
      16 & The last 4 digits are divisible by 16. \\
      \hline
    \end{tabular}

\end{document}
